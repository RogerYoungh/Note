\chapter{线性方程组的解系}

\section{\texorpdfstring{$n$ 维向量空间 $K^n$}{n 维向量空间 Kn}}

取定一个数域 $K$,设 $n$ 是任意给定的一个正整数。令
$$K^n=\{(a_1,\cdots,a_n) \mid a_i\in K,i=1,\cdots,n\}$$
如果 $a_1=b_1,\cdots,a_n=b_n$,则称 $K^n$ 中的两个元素:$(a_1,\cdots,a_n),(b_1,\cdots,b_n)$ 相等。

在 $K^n$ 中规定加法运算:
$$(a_1,\cdots,a_n)+(b_1,\cdots,b_n):=(a_1+b_1,\cdots,a_n+b_n)$$

在 $K$ 的元素与 $K^n$ 的元素之间规定数量乘法运算:
$$k(a_1,\cdots,a_n) := (ka_1,\cdots,ka_n)$$

容易验证加法和数量乘法运算满足下述八条运算法则:对于 $\aaa,\bbb,\ggg\in K^n,k,l\in K$ 有

1. $\aaa+\bbb=\bbb+\aaa$

2. $(\aaa+\bbb)+\ggg=\aaa+(\bbb+\ggg)$

3. 把元素 $(0,\cdots,0)$ 记作零元素 $\ling$,使得
$$\ling + \aaa = \aaa + \ling = \aaa$$

4. 对于 $\aaa = (a_1,\cdots,a_n)\in K^n$,定义其负元素
$$-\aaa := (-a_1,\cdots,-a_n)$$
于是有
$$\aaa + (-\aaa) = (-\aaa)+\aaa = 0$$

5. $1\aaa = \aaa$

6. $(kl)\aaa = k(l\aaa)$

7. $(k+l)\aaa = k\aaa + l\aaa$

8. $k (\aaa+\bbb) = k\aaa + k\bbb$

\begin{definition}[$n$ 维向量空间]
	数域 $K$ 上所有 $n$ 元有序数组组成的集合 $K^n$,连同定义在它上面的加法运算和数量乘法运算,及其满足的 8 条运算法则一起,称为数域 $K$ 上的一个 $n$ 维向量空间。$K^n$ 的元素称为 $n$ 维向量;设向量 $\aaa  = (a_1,\cdots,a_n)$,称 $a_i$ 是 $\aaa$ 的第 $i$ 个分量。
\end{definition}

在 $n$ 维向量空间 $K^n$ 中,可以定义减法运算
$$\aaa - \bbb := \aaa + (-\bbb)$$

$n$ 元有序数组写成一行,称为行向量;写成一列,称为列向量,也可以看作行向量的转置。

$K^n$ 可以看成是 $n$ 维行向量组成的向量空间,也可以看作是列向量组成的向量空间。

\begin{definition}[线性组合]
	给定向量组 $\aaa_1,\cdots,\aaa_s$,再任给 $K$ 中的一组数 $k_1,\cdots,k_s$,那么向量
	$$k_1 \aaa_1+\cdots+k_s \aaa_s$$
	称为向量组 $k_1,\cdots,k_s$ 的一个线性组合,其中 $k_1,\cdots,k_s$ 称为系数。
\end{definition}

\begin{definition}[线性表出]
	给定向量组 $\aaa_1,\cdots,\aaa_s$,对于 $\bbb \in K^n$,若存在 $K$ 中的一组数 $k_1,\cdots,k_s$ 满足
	$$\bbb = k_1\aaa_1+\cdots+k_s\aaa_s$$
	那么称 $\bbb$ 可以由向量组 $\aaa_1,\cdots,\aaa_s$ 线性表出。
\end{definition}

于是可以把数域 $K$ 上的 $n$ 元线性方程组

\begin{equation*}
	\left\{
		\begin{matrix}
			a_{11}x_1+a_{12}x_2+\cdots+a_{1n}x_n=b_1\\
			a_{21}x_1+a_{22}x_2+\cdots+a_{2n}x_n=b_2\\
			\cdots\qquad\cdots\qquad\cdots\\
			a_{n1}x_1+a_{n2}x_2+\cdot +a_{nn}x_n=b_n
		\end{matrix}
	\right.
\end{equation*}

写成
$$x_1\aaa_1+\cdots+x_n\aaa_n=\bbb$$
其中 $\aaa_1,\cdots,\aaa_n$ 是线性方程组的列向量组,$\bbb$ 是由常数项组成的列向量。

\begin{definition}[线性子空间]
	$K^n$ 的一个非空子集 $U$ 是 $K^n$ 的一个线性子空间,那么满足
	
	1. $U$ 对于 $K^n$ 的加法封闭:$\aaa,\ggg\in U \Rightarrow \aaa+\ggg \in U$

	2. $U$ 对于 $K^n$ 的乘法封闭:$\aaa \in U,k\in K \Rightarrow k\aaa \in U$
\end{definition}

特殊的,${0}$ 也是 $K^n$ 的一个,称为零子空间。$K^n$ 本身也是 $K^n$ 的一个子空间。

$\aaa_1,\cdots,\aaa_n$ 的所有线性组合也是 $K^n$ 的一个子空间,称为 $\aaa_1,\cdots,\aaa_n$ 生成(张成)的子空间,记作
$$\langle \aaa_1,\cdots,\aaa_n\rangle:=\{k_1\aaa_1+\cdots+k_s\aaa_s \mid k_i\in K,i=1,\cdots,s\}$$

于是线性方程组有解,等价与 $\bbb$ 可以由 $\aaa_1,\cdots,\aaa_n$ 线性表出,即 $\bbb \in \langle\aaa_1,\cdots,\aaa_n\rangle$。

\section{线性相关与无关}

\begin{definition}
	$K^n$ 中向量组 $\aaa_1,\cdots,\aaa_s$ 称为是线性相关的,如果有 $K$ 中不全为 $0$ 的数 $k_1,\cdots,k_s$,使得
	$$k_1\aaa_1+\cdots+k_s\aaa_s=\ling$$
	否则称为线性无关。
\end{definition}

即线性无关意味着所有的系数只能都为 $0$。

注意线性相关不意味着每个向量都可以由其他向量线性表出,该向量前的 $k$ 可以为 $0$。

\section{向量组的秩}

\begin{definition}[极大线性无关组]
	向量组的一个部分组称为一个极大线性无关组,如果这个部分组本身是线性无关的,但是从这个向量组的其余向量(如果还有的话)中任取一个添进去,得到的新的部分组都线性相关。
\end{definition}

如果向量组 $\aaa_1,\cdots,\aaa_s$ 的每一个向量都可以由向量组 $\bbb_1,\cdots,\bbb_r$ 线性表出,那么称向量组 $\aaa_1,\cdots,\aaa_s$ 可以由向量组线性表出。

\begin{definition}
	如果向量组 $\aaa_1,\cdots,\aaa_s$ 与向量组 $\bbb_1,\cdots,\bbb_r$ 可以互相线性表出,那么称两个向量组等价,记作
	$$\{\aaa_1,\cdots,\aaa_s\} \cong \{\bbb_1,\cdots,\bbb_r\}$$
\end{definition}

可以证明,这种关系具有三条性质(反身性,对称性,传递性),即是等价关系。

对矩阵作初等行变换,变换前后的行向量组等价,不保证列向量组等价。

那么向量组与它的极大线性无关组等价。

\begin{definition}
	向量组 $\aaa_1,\cdots,\aaa_r$ 的极大线性无关组所含向量的个数称为这个向量组的秩,记作
	$$\rank\{\aaa_1,\cdots,\aaa_r\}$$
\end{definition}

\section{子空间的基与维数}

\begin{definition}
	设 $U$ 是 $K^n$ 的一个子空间,如果 $\aaa_1,\cdots,\aaa_r\in U$ 是 $U$ 的一个基,那么

	1. $\aaa_1,\cdots,\aaa_r$ 线性无关。

	2. $U$ 中每一个向量都可以由 $\aaa_1,\cdots,\aaa_r$ 线性表出。
\end{definition}

显然,单位向量组 $\varepsilon_1,\cdots,\varepsilon_n$ 是 $K^n$ 的一个基,称作标准基。

\begin{theorem}
	$K^n$ 的任一非零子空间 $U$ 都有一个基。
\end{theorem}

\begin{theorem}
	$K^n$ 的任一非零子空间 $U$ 的任一两个基所含向量的个数相等,称为 $U$ 的维数,记作 $\dim_KU$ 或 $\dim U$。
\end{theorem}

\begin{theorem}
	向量组 $\aaa_1,\cdots,\aaa_s$ 的一个极大线性无关组是这个向量组生成的子空间的 $\langle \aaa_1,\cdots,\aaa_s \rangle$,从而
	$$\dim\langle \aaa_1,\cdots,\aaa_s \rangle = \rank\{\aaa_1,\cdots,\aaa_s\}$$
\end{theorem}

\section{矩阵的秩}

\begin{theorem}
	阶梯形矩阵 $J$ 的行秩与列秩相等,它们都等于 $J$ 的非零行的个数;并且 $J$ 的主元所在的列构成列向量的一个极大线性无关组。
\end{theorem}

\begin{theorem}
	矩阵的初等行变换不改变矩阵的行秩和列秩。
\end{theorem}

\begin{theorem}
	矩阵的行秩和列秩相等,统称为矩阵的秩。矩阵 $A$ 的秩记作 $\rank(A)$。
\end{theorem}

\begin{theorem}
	非零矩阵的秩等于它的不为零的子式的阶数。
\end{theorem}

若一个 $n$ 级矩阵的秩如果等于它的级数,那么称为满秩矩阵。

\section{线性方程组有解的充分必要条件}

\begin{theorem}
	数域 $K$ 上有线性方程组
	$$x_1\aaa_1 + \cdots + x_n\aaa_n = \beta$$
	有解的充分必要条件是:它的系数矩阵与增广矩阵的秩相等。
\end{theorem}

\begin{theorem}
	数域 $K$ 上 $n$ 元线性方程组有解时,如果它的系数矩阵满秩,那么方程组有唯一解;否则方程组有无穷多个解。
\end{theorem}

\section{齐次线性方程组解集的结构}

数域 $K$ 上 $n$ 元齐次线性方程组
$$x_1\aaa_1 + \cdots + x_n\aaa_n = 0$$
的一个解是 $K^n$ 中的一个向量,称它为齐次线性方程组的一个解向量。

可知齐次线性方程组的解集 $W$ 是 $K^n$ 的一个子空间,称为方程组的一个解空间。

\begin{definition}
	齐次线性方程组有非零解时,如果它的有限多个解 $\eee_1,\cdots,\eee_t$ 是其基础解系

	1. $\eee_1,\cdots,\eee_t$ 线性无关。

	2. 齐次线性方程组的每一个解都可以由 $\eee_1,\cdots,\eee_t$ 线性表出。
\end{definition}

于是解空间为
$$W = \langle \eee_1,\cdots,\eee_t \rangle$$

\begin{theorem}
	数域 $K$ 上 $n$ 元齐次线性方程组的解空间 $W$ 的维数为
	$$\dim W = n-\rank(A)$$
\end{theorem}

\section{非齐次线性方程组的结构}

称数域 $K$ 上 $n$ 元齐次线性方程组
$$x_1\aaa_1 + \cdots + x_n\aaa_n = \bbb$$
的导出组为
$$x_1\aaa_1 + \cdots + x_n\aaa_n = 0$$
其的解空间用 $W$ 表示。

\begin{theorem}
	如果数域 $K$ 上 $n$ 元齐次线性方程组有解,那么它的解集 $U$ 为
	$$U = \{\ggg_0+\eee \mid \eee \in W\}$$
	其中 $\ggg_0$ 是非其次线性方程组的一个特解,$W$ 是导出组的解空间用。
\end{theorem}









