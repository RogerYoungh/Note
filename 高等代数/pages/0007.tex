\chapter{多项式环}

\section{一元多项式环}

\begin{definition}
    数域 $K$ 上的一元多项式是指如下述的表达式
    $$a_nx^n+\cdots + a_1x + a_0$$
    其中 $x$ 是一个符号(它不属于 $K$)称为不定元,$n$ 是非负整数,$a_i\in K(i=0,\cdots,n)$ 称为系数,$a_ix^i$ 称为 $i$ 次项($i=1,\cdots,n$),$a_0$ 称为零次项或常数项。
\end{definition}

两个这种形式的表达式相等规定为它们含有完全相同的项。此时符号 $x$ 为不定元。系数全为 $0$ 的多项式称为零多项式,记作 $0$。

因此两个一元多项式相等当且仅当它们的同次项都对应相等相等,即一元多项式 的表示方式是唯一的。

我们常常用 $f(x),g(x),h(x),\cdots$ 或 $f,g,h,\cdots$ 表示一元多项式。

一元多项式的重要特点是它有次数概念。设
$$f(x) = a_nx^n+\cdots + a_1x + a_0$$
如果 $a_n\ne 0$,那么称 $a_nx^n$ 是 $f(x)$ 的首项,称 $n$ 是 $f(x)$ 的次数,记作 $\deg f(x)$。

零多项式的次数定义为 $-\infty$,并且规定对于任意 $n\in \NN$
$$(-\infty)+(-\infty):=-\infty$$
$$(-\infty) +n := -\infty$$
$$-\infty < n$$

数域 $K$ 上所有一元多项式组成的集合记作 $K[x]$。在 $K[x]$ 中可以定义加法和乘法运算。

设 $f(x) = \displaystyle\sum_{i=0}^na_ix^i,g(x) = \sum_{i=0}^mb_ix^i$,不妨设 $m \leqslant n$,令
$$f(x) + g(x) := \sum_{i=0}^n(a_i+b_i)x^i$$
$$f(x)g(x) := \sum_{s=0}^{m+n}\left(\sum_{i+j=s}a_ib_j\right)$$

容易验证,一元多项式 的加法满足交换律、结合律。同样,其适合消去律
$$f(x)g(x) = f(x)h(x),\text{且}\ f(x)\ne 0 \Rightarrow g(x) = h(x)$$

\subsection{环的基本概念}

集合 $S$ 上的一个代数运算,是指 $S\times S$ 到 $S$ 的一个映射。

\begin{definition}
    设 $R$ 是一个非空集合,如果其上定义了加法和乘法两个代数运算,并且满足如下六条运算法则,其中 $\forall a,b,c\in R$:

    1. 加法结合律 $(a+b)+c = a+(b+c)$。

    2. 加法交换律 $a+b = b+a$。

    3. 在 $R$ 中有元素 $0$,使得 $a+0=a$,称 $0$ 为 $R$ 的零元素。

    4. 对于 $a$,在 $R$ 中有元素 $d$,使得 $a+d=0$,称 $d$ 是 $a$ 的负元素,记作 $-a$。

    5. 乘法结合律 $(ab)c = a(bc)$。

    6. 乘法对于加法的左、右分配律
    $$a(b+c) = ab+ac,(b+c)a = ba+ca$$

    那么称 $R$ 是一个环。
\end{definition}

\begin{theorem}
    环 $R$ 中的零元素是唯一的,元素 $a$ 的负元素是唯一的。
\end{theorem}
\begin{proof}
    hhh 
\end{proof}