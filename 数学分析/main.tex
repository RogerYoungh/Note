\documentclass[cn,11pt,blue,normal,founder]{elegantbook}
\usepackage{amsmath}
\usepackage{amsfonts}
\usepackage{amssymb}
\usepackage{lmodern}

\begin{document}
\newcommand\mfrac[2]{\dfrac{#1\smash[b]{\strut}}{#2\smash[t]{\strut}}}
\newcommand\RR{\mathbb{R}}
\newcommand\NN{\mathbb{N}}
\newcommand\QQ{\mathbb{Q}}
\newcommand\ee{\mathrm{e}}
\newcommand\dd{\mathrm{d}}
\newcommand\uppi{\mathrm{\pi}}
%%----------- 封面部分 ----------- %%

\begin{titlepage}
	\vspace*{25ex}% A4->35,Boox->25,Kindle->10
	\begin{minipage}{.9\textwidth}
	\flushright
		{\zihao{0}\textbf{数学分析笔记}}\\%%中文书名
		\rule{\linewidth}{1pt}\\ \vspace{2ex}
		{\zihao{2}\textsf{ShuXueFenXi}} \\%% 英文书名
		\vspace{20ex}% A4->35,Boox->20,Kindle->10
		{\zihao{4}rogeryoungh}%% 作者
	\end{minipage}
	\vfill\centering
	{\zihao{4}\number\year 年 \number\month 月 \number\day 日 · \LaTeX{}}
\end{titlepage}
%%----------- 目录部分 ----------- %%
\clearpage{\hypersetup{hidelinks}\tableofcontents}

%%----------- 主体部分 ----------- %%
\clearpage

\chapter{线性方程组的解法}

\section{矩阵消元法}

形如这样左端都是未知量 $x_n$ 的一次齐次式,右端是常数,
$$a_1x_1+a_2x_2+\cdots+a_nx_n=b$$
像这样的方程称为线性方程。每个未知量前面的数称为系数,右端的项称为常数项。

含 $n$ 个未知量的线性方程组称为 $n$ 元线性方程组,它的一般形式是
\begin{equation*}
	\left\{
		\begin{matrix}
			a_{11}x_1+a_{12}x_2+\cdots+a_{1n}x_n=b_1\\
			a_{21}x_1+a_{22}x_2+\cdots+a_{2n}x_n=b_2\\
			\cdots\qquad\cdots\qquad\cdots\\
			a_{s1}x_1+a_{s2}x_2+\cdots+a_{sn}x_n=b_s
		\end{matrix}
	\right.
\end{equation*}
方程的个数 $s$ 与未知量的个数 $n$ 可以相等,也可以不等。

\begin{definition}[线性方程组的初等变换]
	线性方程组的初等变换有三种,分别为:
	
	1. 把一个方程的倍数加到另一个方程上。

	2. 互换两个方程的位置。

	3. 用一个非零数乘某一个方程。
\end{definition}

对于线性方程组,若 $x_1,x_2,\cdots,x_n$ 分别可以用数 $c_1,c_2,\cdots,c_n$ 代入后,每个方程都变成恒等式,那么称 $n$ 元有序组 $(c_1,c_2,\cdots,c_n)$ 是线性方程组的一个解。方程组所有解组成的集合称为这个线性方程组的解集,符合实际要求的解称为可行解。

通过初等变换能够使线性方程组变为阶梯形方程组,进一步可以变为简化阶梯形方程组,此种形式可以较方便的看出方程组的解。

\begin{theorem}
	初等变换不改变线性方程组的解。
\end{theorem}

可以把原线性方程组的系数和常数项按次序排成一张表,称为方程组的增广矩阵;而只列出系数的方程组称为系数矩阵。

\begin{definition}
	由 $sn$ 个数排成的 $s$ 行(横的)$n$ 列(纵的)表
	\begin{equation*}
		\left(
			\begin{matrix}
				a_{11}&a_{12}&\ldots&a_{1n}\\
				a_{21}&a_{22}&\ldots&a_{2n}\\
				\vdots&\vdots&&\vdots\\a_{s1}&a_{s2}&\ldots&a_{sn}\\
			\end{matrix}
		\right)
	\end{equation*}
	称为一个 $s\times n$ 矩阵,记作 $A_{s\times n}$ 或 $A=(a_{ij})$,它的 $(i,j)$ 元也记作 $A(i;j)$。
\end{definition}

特殊的,如果矩阵 $A$ 的行数和列数相等皆为 $n$,则称它为 $n$ 级方阵或方阵。元素全为 $0$ 的矩阵称为零矩阵,记作 $0_{s\times n}$ 或 $0$。

\begin{definition}[初等行变换]
	矩阵的初等行变换有三种,分别为:
	
	1. 把一行的倍数加到另一行上。

	2. 互换两行的位置。

	3. 用一个非零数乘某一行。
\end{definition}

矩阵经过初等行变换,可变成阶梯形矩阵,并可进一步化简成简化行阶梯形矩阵。

阶梯形矩阵的特点为(1)元素全为 $0$ 的行(零行)在下方(如果有的话);(2)元素不全为 $0$ 的行(非零行),左起第一个不为 $0$ 的元素(主元),他们的列指标随着行指标递增而严格增大。

有些书上有行最简形矩阵,它的特点为(1)它是阶梯形矩阵;(2)每个主元所在的列的其余元素都是 $0$。

简化行阶梯形矩阵的特点为(1)它是阶梯形矩阵;(2)每个非零行的主元都是 $1$;(3)每个主元所在的列的其余元素都是 $0$。

在解线性方程组时,可以通过一系列初等行变换,它的增广矩阵化为阶梯形矩阵,甚至继续化简为简化行阶梯形矩阵,都可简化求解过程。

\begin{theorem}
	任意矩阵都可以经过一系列初等行变换化为阶梯形矩阵,也可以变成简化行阶梯形矩阵。
\end{theorem}

\section{线性方程组的解的情况及其判别准则}


由于初等变换不改变线性方程组的解,其总可以化为阶梯形方程组。因此设阶梯形方程组有 $n$ 个未知量,它的增广矩阵 $J$ 有 $r$ 个非零行,$J$ 有 $n+1$ 列。

1. 若阶梯形方程组中出现 $0=d$(其中 $d$ 为非零数)这种方程,即最后一个非零行的主元位于 $n+1$ 列,则阶梯形方程组无解。

2. 最后一个非零行的主元不位于 $n+1$ 列。

2(1). $r=n$ 时,阶梯形方程恰有唯一解。

2(2). $r<n$ 时,有无穷多组解。

\begin{theorem}
	系数为有理数(实数、复数)的 $n$ 元线性方程组的解的情况只有三种可能:无解,有唯一解,有无穷多组解。
\end{theorem}

若一个线性方程组有解,则称它是相容的;否则称它是不相容的。

\section{数域}

\begin{definition}
	复数集的一个子集 $K$ 是一个数域,那么满足:
	
	1. $0,1\in K$;
	
	2. $a,b\in K \Rightarrow a \pm b,ab\in K$;
	
	3. $a,b \in K$,且 $b\ne 0 \Rightarrow \dfrac{a}{b}\in K$。
\end{definition}

其中,$\QQ,\RR,\CC$ 都是数域,但整数集 $\mathrm{Z}$ 不是数域。

有理数域是最小的数域。

\begin{theorem}
	任意数域都包含有理数域。
\end{theorem}


 % 实数集与函数
\chapter{行列式}

\section{排列}

\begin{definition}
	由 $1,2,\cdots,n$ 组成的一个有序数组称为一个 $n$ 阶排列。
\end{definition}

特殊的,排列 $12\cdots n$ 也是一个 $n$ 阶排列,称为自然排列。

\begin{definition}
	在一个排列中,如果一对数前面的数大于后面的数,那么它们就称为一个逆序,反之称为正序。
	
	排列中逆序的对数称为这个排列的逆序数。
\end{definition}

逆序数为偶数的排列称为偶排列,逆序数的排列称为奇排列。

排列 $j_1j_2\cdots j_n$ 的逆序数记为
$$\tau(j_1j_2\cdots j_n)$$

\begin{definition}
	把排列中某两个数位置互换,得到一个新排列。称这样的一个变换称为一个对换。
\end{definition}

\begin{theorem}
	对换改变排列的奇偶性。
\end{theorem}

\begin{proof}
	设对换为 $(i,j)$,分类讨论:

	1. 若所换两数相邻,则不影响后面数字的逆序数。那么若 $ij$ 为一个逆序,则逆序数减 $1$,否则逆序数加 $1$,总之逆序数奇偶性改变。

	2. 若所换两数不相邻,不妨设 $i$ 在 $j$ 之前,排列为
	$$\cdots i \quad k_1 \cdots k_s \quad j\cdots$$
	那么有
	$$(i,j)=(i,k_1)\cdots(i,k_s)(i,j)(k_s,j)\cdots(k_1,j)$$
	即任意对换即总可以分解为奇数个相邻对换的积。
\end{proof}

任何一个 $n$ 阶排列都可以与自然排列由一系列对换互变,即置换。奇置换可以分解为奇数个对换的积,偶置换可以分解为偶数个对换的积。

\section{\texorpdfstring{$n$ 阶行列式}{n 阶行列式}}

记 $\displaystyle\sum_{j_1j_2\cdots j_n}$ 表示对所有 $n$ 元排列求和。

\begin{definition}
	$n$ 阶行列式
	\begin{equation*}
		\left|\begin{matrix}
			a_{11}&a_{12}&\ldots&a_{1n}\\
			a_{21}&a_{22}&\ldots&a_{2n}\\
			\vdots&\vdots&&\vdots\\
			a_{n1}&a_{n2}&\ldots&a_{nn}
		\end{matrix}\right|
		 = \sum_{j_1j_2\cdots j_n}(-1)^{\tau(j_1j_2\cdots j_n)}a_{1j_1}a_{2j_2}\cdots a_{nj_n}
	\end{equation*}
	该式称为 $n$ 阶行列式的完全展开式。
\end{definition}

若对角线下方的元素全为 $0$,则称为上三角行列式,即对于所有的 $1\leqslant i < j\leqslant n$,有 $a_{ji}=0$。

\begin{theorem}
	上三角行列式的值为
	$$|A| = a_{11}a_{22}\cdots a_{nn}$$
\end{theorem}

\begin{proof}
	考虑其完整展开式的任意一项
	$$(-1)^{\tau(j_1j_2\cdots j_n)}a_{1j_1}a_{2j_2}\cdots a_{nj_n}$$
	若该项不为 $0$,则对 $1 \leqslant k \leqslant n$,皆有 $j_k\leqslant k$。
	
	因此只有 $j_k=k$,只有这一项不为 $0$。
\end{proof}

\begin{theorem}
	给定行指标的一个排列 $i_1i_2\cdots i_n$,则 $n$ 级矩阵的行列式为
	$$|A| = \sum_{k_1k_2\cdots k_n}(-1)^{\tau(i_1i_2\cdots i_n)+\tau(k_1k_2\cdots k_n)}a_{i_1k_1}a_{i_2k_2}\cdots a_{i_nk_n}$$
\end{theorem}

\begin{proof}
	设 $a_{1j_1}a_{2j_2}\cdots a_{nj_n}$ 经过 $s$ 次互换相邻元素变为 $a_{i_1k_1}a_{i_2k_2}\cdots a_{i_nk_n}$,则有
	\begin{equation*}
		\begin{aligned}
			(-1)^{\tau(i_1i_2\cdots i_n)}&=(-1)^s\\
			(-1)^{\tau(j_1j_2\cdots j_n)}(-1)^s&=(-1)^{\tau(k_1k_2\cdots k_n)}
		\end{aligned}
	\end{equation*}
	从而
	\begin{equation*}
		\begin{aligned}
			(-1)^{\tau(i_1i_2\cdots i_n)+\tau(k_1k_2\cdots k_n)}&=(-1)^s(-1)^{\tau(j_1j_2\cdots j_n)}(-1)^s\\
			&=(-1)^{\tau(j_1j_2\cdots j_n)}
		\end{aligned}
	\end{equation*}
\end{proof}

同理,给定列指标的一个排列 $k_1k_2\cdots k_n$,则 $n$ 级矩阵的行列式为
$$|A| = \sum_{i_1i_2\cdots i_n}(-1)^{\tau(i_1i_2\cdots i_n)+\tau(k_1k_2\cdots k_n)}a_{i_1k_1}a_{i_2k_2}\cdots a_{i_nk_n}$$

\section{行列式的性质}

\begin{theorem}
	转置后行列式值不变。
\end{theorem}

记矩阵 $A$ 转置后的矩阵 $A'$ 或 $A^T$。

\begin{proof}
	设行列式 $B=A'$,即 $a_{ij}=b_{ji}$。由前文知,按列指标展开 $B$ 有(注意第 $1$ 个下标是列指标,第 $2$ 个下标是行指标)
	$$|B|=\sum_{i_1i_2\cdots i_n}(-1)^{\tau(i_1i_2\cdots i_n)}a_{1i_1}a_{2i_2}\cdots a_{ni_n}$$
	按列指标展开 $A$ 有
	$$|A|=\sum_{i_1i_2\cdots i_n}(-1)^{\tau(i_1i_2\cdots i_n)}a_{1i_1}a_{2i_2}\cdots a_{ni_n}$$
	因此 $|A|=|B|$。
\end{proof}

\begin{theorem}
	行列式的一行乘以一个数,等于行列式乘以这个数。
\end{theorem}

\begin{proof}
	即行列式 $B$ 除了第 $i_1$ 行有 $b_{i_1j}=ka_{i_1j}$,其他行都有 $b_{ij}=a_{ij}$。
	\begin{equation*}
		\begin{aligned}
			|B| &= \sum_{j_1j_2\cdots j_n}(-1)^{\tau(j_1j_2\cdots j_n)}a_{1j_1}\cdots (ka_{i_1j}) \cdots a_{nj_n}\\
			&= k\sum_{j_1j_2\cdots j_n}(-1)^{\tau(j_1j_2\cdots j_n)}a_{1j_1}\cdots a_{i_1j} \cdots a_{nj_n}\\
			&=k|A|
		\end{aligned}
	\end{equation*}
\end{proof}

\begin{theorem}
	除了同一行以外全部相等的两个行列式,与此行替换为这两行的和的行列式相等。
\end{theorem}

\begin{proof}
	即行列式 $A$ 除了第 $i_1$ 行有 $a_{i_1j}=b_{i_1j}+c_{i_1j}$,其他行都有 $a_{ij}=b_{ij}=c_{ij}$。
	\begin{equation*}
		\begin{aligned}
			|A| &= \sum_{j_1j_2\cdots j_n}(-1)^{\tau(j_1j_2\cdots j_n)}a_{1j_1}\cdots (b_{i_1j}+c_{i_1j}) \cdots a_{nj_n}\\
			&= \sum_{j_1j_2\cdots j_n}(-1)^{\tau(j_1j_2\cdots j_n)}a_{1j_1}\cdots b_{i_1j} \cdots a_{nj_n}+\sum_{j_1j_2\cdots j_n}(-1)^{\tau(j_1j_2\cdots j_n)}a_{1j_1}\cdots c_{i_1j} \cdots a_{nj_n}\\
			&=|B|+|C|
		\end{aligned}
	\end{equation*}
\end{proof}

\begin{theorem}
	行列式中有两行互换,行列式反号。
\end{theorem}

\begin{proof}
	即设行列式 $B$ 为行列式第 $k_1,k_2$ 两行交换的结果,又
	$$(-1)^{\tau(j_1\cdots j_{k_2} \cdots j_{k_1} \cdots j_n)} =- (-1)^{\tau(j_1\cdots j_{k_1} \cdots j_{k_2} \cdots j_n)}$$
	那么有
	\begin{equation*}
		\begin{aligned}
			|B| &= \sum_{j_1\cdots j_{k_2} \cdots j_{k_1} \cdots j_n}(-1)^{\tau(j_1\cdots j_{k_2} \cdots j_{k_1} \cdots j_n)}
			a_{1j_1}\cdots a_{k_1j_{k_2}}\cdots a_{k_2j_{k_1}}\cdots a_{nj_n}\\
			&=-\sum_{j_1\cdots j_{k_1} \cdots j_{k_2} \cdots j_n}(-1)^{\tau(j_1\cdots j_{k_1} \cdots j_{k_2} \cdots j_n)}
			a_{1j_1}\cdots a_{k_1j_{k_1}}\cdots a_{k_2j_{k_2}}\cdots a_{nj_n}\\
			&=-|A|
		\end{aligned}
	\end{equation*}
\end{proof}

\begin{theorem}
	行列式中有两行相等,行列式为零。
\end{theorem}

\begin{proof}
	交换这相同的两行,行列式变号,其仍与原来相等,只能为 $0$。
\end{proof}

\begin{theorem}
	行列式中两行成比例,行列式为零。
\end{theorem}

\begin{proof}
	提出公因子使两行相等,即为 $0$。
\end{proof}

\begin{theorem}
	把一行的倍数加到另一行,行列式不变。
\end{theorem}

\begin{proof}
	一行是另一行的倍数的行列式为 $0$,合并后自然不变。
\end{proof}

\section{行列式按一行展开}

\begin{definition}[代数余子式]
	$n$ 阶行列式 $|A|$ 中,划去第 $i$ 行和第 $j$ 列,剩下的元素按原来次序组成的 $n-1$ 阶行列式称为矩阵 $A$ 的 $(i,j)$ 元的余子式。记作 
	$$M_{ij} = \sum_{k_1\cdots k_{i-1}k_{i+1}\cdots k_n}(-1)^{\tau(k_1\cdots k_{i-1}k_{i+1}\cdots k_n)}a_{1k_1}\cdots a_{i-1,k_{i-1}}a_{i+1,k_{i+1}}\cdots a_{nk_n}$$
	其中 $j=k_i$,令 $A_{ij}=(-1)^{i+j}M_{ij}$,称 $A_{ij}$ 是 $A$ 的 $(i,j)$ 元的代数余子式。
\end{definition}

\begin{theorem}
	对于 $n$ 阶行列式 $|A|$ 有
	$$|A| = \sum_{j=1}^na_{ij}A_{ij} = \sum_{i=1}^na_{ij}A_{ij}$$
	前者称为 $n$ 阶行列式按第 $i$ 行的展开式,后者称为按第 $j$ 列的展开式。
\end{theorem}

\begin{proof}
	首先列出 $|A|$ 的行完全展开式,其中 $j=k_i$
	$$|A|=\sum_{k_1\cdots k_{i-1}jk_{i+1}\cdots k_n}(-1)^{\tau(k_1\cdots k_{i-1}jk_{i+1}\cdots k_n)}a_{1k_1}\cdots a_{i-1,k_{i-1}}a_{ij}a_{i+1,k_{i+1}}\cdots a_{nk_n}$$
	把第 $i$ 行换到第 $1$ 行,第 $j$ 列换到第 $1$ 列,由对换的性质有
	$$(-1)^{\tau(i1\cdots (i-1)(i+1)\cdots n)+\tau(jk_1\cdots k_{i-1}k_{i+1}\cdots k_n)}=(-1)^{i-1}(-1)^{j-1}(-1)^{\tau(k_1\cdots k_{i-1}k_{i+1}\cdots k_n)}$$
	因此
	\begin{equation*}
		\begin{aligned}
			|A|&=\sum_{j=1}^na_{ij}(-1)^{i+j}\sum_{k_1\cdots k_{i-1}k_{i+1}\cdots k_n}(-1)^{\tau(k_1\cdots k_{i-1}k_{i+1}\cdots k_n)}a_{1k_1}\cdots a_{i-1,k_{i-1}}a_{i+1,k_{i+1}}\cdots a_{nk_n}\\
			&=\sum_{j=1}^na_{ij}(-1)^{i+j}M_{ij}=\sum_{j=1}^na_{ij}A_{ij}
		\end{aligned}
	\end{equation*}
	对于列展开式,转置即可。
\end{proof}

\begin{theorem}
	对于 $n$ 阶行列式 $|A|$ 有
	$$\sum_{j=1}^na_{ij}A_{kj} = 0(k\ne i)$$
\end{theorem}

\begin{proof}
	设矩阵 $B$ 第 $k$ 行与第 $i$ 行相等,因此按第 $k$ 行展开有
	$$|B|=\sum_{j=1}^na_{kj}A_{kj}=\sum_{j=1}^na_{ij}A_{kj}=0$$
\end{proof}

\begin{definition}[范德蒙 Vandermonde 行列式]
	 若行列式满足 $a_{ij} = a_j^i$,则称为范德蒙特行列式。其值为(证略)
	 $$\prod_{1\leqslant j < i \leqslant n}(a_i-a_j)$$
\end{definition}

\section{克莱姆(Cramer)法则}

对于数域 $K$ 上 $n$ 个方程的 $n$ 元线性方程组,

\begin{equation*}
	\left\{
		\begin{matrix}
			a_{11}x_1+a_{12}x_2+\cdots+a_{1n}x_n=b_1\\
			a_{21}x_1+a_{22}x_2+\cdots+a_{2n}x_n=b_2\\
			\cdots\qquad\cdots\qquad\cdots\\
			a_{n1}x_1+a_{n2}x_2+\cdot +a_{nn}x_n=b_n
		\end{matrix}
	\right.
\end{equation*}

其系数矩阵记作 $A$,增广矩阵记作 $\widetilde{A}$  % 数列极限
\chapter{线性方程组的解系}

\section{\texorpdfstring{$n$ 维向量空间 $K^n$}{n 维向量空间 Kn}}

取定一个数域 $K$,设 $n$ 是任意给定的一个正整数。令
$$K^n=\{(a_1,\cdots,a_n) \mid a_i\in K,i=1,\cdots,n\}$$
如果 $a_1=b_1,\cdots,a_n=b_n$,则称 $K^n$ 中的两个元素:$(a_1,\cdots,a_n),(b_1,\cdots,b_n)$ 相等。

在 $K^n$ 中规定加法运算:
$$(a_1,\cdots,a_n)+(b_1,\cdots,b_n):=(a_1+b_1,\cdots,a_n+b_n)$$

在 $K$ 的元素与 $K^n$ 的元素之间规定数量乘法运算:
$$k(a_1,\cdots,a_n) := (ka_1,\cdots,ka_n)$$

容易验证加法和数量乘法运算满足下述八条运算法则:对于 $\aaa,\bbb,\ggg\in K^n,k,l\in K$ 有

1. $\aaa+\bbb=\bbb+\aaa$

2. $(\aaa+\bbb)+\ggg=\aaa+(\bbb+\ggg)$

3. 把元素 $(0,\cdots,0)$ 记作零元素 $\ling$,使得
$$\ling + \aaa = \aaa + \ling = \aaa$$

4. 对于 $\aaa = (a_1,\cdots,a_n)\in K^n$,定义其负元素
$$-\aaa := (-a_1,\cdots,-a_n)$$
于是有
$$\aaa + (-\aaa) = (-\aaa)+\aaa = 0$$

5. $1\aaa = \aaa$

6. $(kl)\aaa = k(l\aaa)$

7. $(k+l)\aaa = k\aaa + l\aaa$

8. $k (\aaa+\bbb) = k\aaa + k\bbb$

\begin{definition}[$n$ 维向量空间]
	数域 $K$ 上所有 $n$ 元有序数组组成的集合 $K^n$,连同定义在它上面的加法运算和数量乘法运算,及其满足的 8 条运算法则一起,称为数域 $K$ 上的一个 $n$ 维向量空间。$K^n$ 的元素称为 $n$ 维向量;设向量 $\aaa  = (a_1,\cdots,a_n)$,称 $a_i$ 是 $\aaa$ 的第 $i$ 个分量。
\end{definition}

在 $n$ 维向量空间 $K^n$ 中,可以定义减法运算
$$\aaa - \bbb := \aaa + (-\bbb)$$

$n$ 元有序数组写成一行,称为行向量;写成一列,称为列向量,也可以看作行向量的转置。

$K^n$ 可以看成是 $n$ 维行向量组成的向量空间,也可以看作是列向量组成的向量空间。

\begin{definition}[线性组合]
	给定向量组 $\aaa_1,\cdots,\aaa_s$,再任给 $K$ 中的一组数 $k_1,\cdots,k_s$,那么向量
	$$k_1 \aaa_1+\cdots+k_s \aaa_s$$
	称为向量组 $k_1,\cdots,k_s$ 的一个线性组合,其中 $k_1,\cdots,k_s$ 称为系数。
\end{definition}

\begin{definition}[线性表出]
	给定向量组 $\aaa_1,\cdots,\aaa_s$,对于 $\bbb \in K^n$,若存在 $K$ 中的一组数 $k_1,\cdots,k_s$ 满足
	$$\bbb = k_1\aaa_1+\cdots+k_s\aaa_s$$
	那么称 $\bbb$ 可以由向量组 $\aaa_1,\cdots,\aaa_s$ 线性表出。
\end{definition}

于是可以把数域 $K$ 上的 $n$ 元线性方程组

\begin{equation*}
	\left\{
		\begin{matrix}
			a_{11}x_1+a_{12}x_2+\cdots+a_{1n}x_n=b_1\\
			a_{21}x_1+a_{22}x_2+\cdots+a_{2n}x_n=b_2\\
			\cdots\qquad\cdots\qquad\cdots\\
			a_{n1}x_1+a_{n2}x_2+\cdot +a_{nn}x_n=b_n
		\end{matrix}
	\right.
\end{equation*}

写成
$$x_1\aaa_1+\cdots+x_n\aaa_n=\bbb$$
其中 $\aaa_1,\cdots,\aaa_n$ 是线性方程组的列向量组,$\bbb$ 是由常数项组成的列向量。

\begin{definition}[线性子空间]
	$K^n$ 的一个非空子集 $U$ 是 $K^n$ 的一个线性子空间,那么满足
	
	1. $U$ 对于 $K^n$ 的加法封闭:$\aaa,\ggg\in U \Rightarrow \aaa+\ggg \in U$

	2. $U$ 对于 $K^n$ 的乘法封闭:$\aaa \in U,k\in K \Rightarrow k\aaa \in U$
\end{definition}

特殊的,${0}$ 也是 $K^n$ 的一个,称为零子空间。$K^n$ 本身也是 $K^n$ 的一个子空间。

$\aaa_1,\cdots,\aaa_n$ 的所有线性组合也是 $K^n$ 的一个子空间,称为 $\aaa_1,\cdots,\aaa_n$ 生成(张成)的子空间,记作
$$\langle \aaa_1,\cdots,\aaa_n\rangle:=\{k_1\aaa_1+\cdots+k_s\aaa_s \mid k_i\in K,i=1,\cdots,s\}$$

于是线性方程组有解,等价与 $\bbb$ 可以由 $\aaa_1,\cdots,\aaa_n$ 线性表出,即 $\bbb \in \langle\aaa_1,\cdots,\aaa_n\rangle$。




 % 函数极限
\chapter{矩阵的运算}

\section{矩阵的运算}

数域上 $K$ 两个矩阵的行数、列数都相等,且所有元素对应相等,那么称两个矩阵相等。

\begin{definition}
    设数域 $K$ 上的 $s\times n$ 矩阵 $A=(a_{ij}),B=(b_{ij})$,令 $A,B$ 的和为
    $$A+B := (a_{ij}+b_{ij})_{s\times n}$$
\end{definition}

\begin{definition}
    设数域 $K$ 上的 $s\times n$ 矩阵 $A=(a_{ij})$,令 $k\in K$ 与 $A$ 的数量乘积为
    $$kA := (ka_{ij})_{s\times n}$$
\end{definition}

容易验证,矩阵的加法和数量乘法满足类似于 $n$ 维向量的 8 条运算法则。

同样定义矩阵的减法
$$A-B := A+(-B)$$

\begin{definition}
    设 $A=(a_{ij})_{s\times n},B=(a_{ij})_{n\times m}$,令 $A,B$ 的乘积为
    $$AB = \left(\sum_{k=1}^na_{ik}b_{kj}\right)_{s\times m}$$
\end{definition}

同样,若 $AB$ 和 $BA$ 几乎都不相等,甚至不一定能够运算。

\begin{theorem}
    设 $A=(a_{ij})_{s\times n},B=(b_{ij})_{n\times m},C=(c_{ij})_{m\times r}$,则
    $$(AB)C = A(BC)$$
\end{theorem}

注意到,若 $A,B\ne 0$,有可能 $BA = 0$。因此 $BA = 0$ 不能推出 $B=0$ 或 $A=0$。

\begin{definition}[零因子]
    对于矩阵 $A$,若存在矩阵 $B\ne 0$ 使得 $AB = 0$,那么称 $A$ 是一个左零因子。
    
    如果存在一个矩阵 $C\ne 0$ 使得 $CA = 0$,那么称 $A$ 是一个右零因子。
    
    左零因子和右零因子称为零因子。 
\end{definition}

特殊的,零矩阵是零因子,称为平凡的零因子。

\begin{theorem}
    矩阵的乘法有分配律(左分配律、右分配律)
    $$A(B+C) = AB+AC$$
    $$(B+C)D = BD + CD$$
\end{theorem}

矩阵的乘法不适合消去律,从 $AC = BC$ 且 $C\ne 0$ 不能推出 $A=B$。

主对角线上元素都是 $1$,其余元素都是 $0$ 的 $n$ 级矩阵称为 $n$ 级单位矩阵,记作 $I_n$ 或者简记作 $I$。主对角线上元素是同一个数 $k$,其余元素全为 $0$ 的 $n$ 级矩阵称为数量矩阵,可以记作 $kI$。一些书上写作 $\mathrm{E}$。

因此有
$$I_s A_{s\times n} = A_{s\times n}, A_{s\times n} I_n= A_{s\times n}$$
若 $A$ 是 $n$ 级矩阵,则
$$IA = AI = A$$
矩阵的乘法与数量乘法满足下述关系式
$$k(AB) = (kA)B = A(kB)$$

数量矩阵还有
$$kI + lI = (k+l)I$$
$$k(lI) = (kl)I$$
$$(kI)(lI) = (kl)I$$

矩阵的乘法虽不满足交换律,但若对具体的两个矩阵 $A$ 与 $B$,也有可能 $AB = BA$,那么称 $A$ 与 $B$ 可交换。比如数量矩阵与任一同级矩阵可交换
$$(kI)A = A(kI) = kA$$

\begin{definition}
    定义 $n$ 级矩阵 $A$ 的非负整数次幂为

    1. $A^n := I$

    2. $A^{m+1} := AA^m$
\end{definition}


\begin{theorem}
    1. $(A+B)' = A'$
    
    2. $(kA)' = kA'$
    
    3. $(AB)' = B'A'$
\end{theorem}

如果把 $n$ 元线性方程组的系数矩阵记作 $A$,称常数项组成的列向量为 $\bbb$,未知量 $x_1,\cdots,x_n$ 组成的列向量为 $\XXX$,那么 $n$ 元线性方程组可以写成
$$A\XXX = \bbb$$
于是列向量 $\eee$ 是方程组的 $A\XXX = \bbb$ 的解当且仅当 $A\eee = \bbb$。

\section{特殊矩阵}

\begin{definition}
    主对角线以外的元素全为 $0$ 的方阵称为对角矩阵,简记作
    $$\diag\{d_1,\cdots,d_n\}$$
\end{definition}

\begin{definition}
    只有一个元素是 $1$,其他元素全为 $0$ 的矩阵称为基本矩阵。$(i,j)$ 元为 $1$ 的基本矩阵记作 $E_{ij}$。
\end{definition}

\begin{definition}
    主对角线下(上)方的元素全为 $0$ 的方阵称为上(下)三角矩阵。
\end{definition}

显然 $A=(a_{ij})$ 为上三角矩阵的 充分必要条件是
$$a_{ij}=0,\text{当}\ i>j$$
同样,上三角矩阵也可表述为
$$A = \sum_{i=1}^n\sum_{j=i}^na_{ij}E_{ij}$$

\begin{definition}
    由单位矩阵经过一次初等行(列)变换得到的矩阵称为初等矩阵。
\end{definition}

初等矩阵有且只有三种类型:$P(j,i(k)),P(i,j),P(i(c))$,其中 $c\ne 0$。

1. 用 $P(j,i(k))$ 左乘 $A$,即把 $A$ 的第 $i$ 行的 $k$ 倍加到第 $j$ 行上。

2. 用 $P(j,i(k))$ 右乘 $A$,即把 $A$ 的第 $j$ 列的 $k$ 倍加到第 $i$ 列上。

类似的,用 $P(i,j)$ 左(右)乘 $A$,就相当于把 $A$ 的第 $i$ 行(列)与第 $j$ 行(列)互换。

用 $P(i(c))$ 左(右)乘 $A$,就相当于用 $c$ 乘 $A$ 的第 $i$ 行(列)。

\begin{definition}
    一个矩阵 $A$ 如果满足 $A' = A$,那么称 $A$ 是对称矩阵。
\end{definition}

\begin{definition}
    如果一个矩阵 $A$ 如果满足 $A' = -A$,那么称 $A$ 是斜对称矩阵。
\end{definition}

\section{矩阵乘积的秩与行列式}

\begin{theorem}
    设 $A = (a_{ij})_{s\times n},B = (b_{ij})_{n\times m}$,则
    $$\rank(AB) \leqslant \min\{\rank(A),\rank(B)\}$$
\end{theorem}

\begin{theorem}
    设 $A = (a_{ij})_{s\times n},B = (b_{ij})_{n\times m}$,则
    $$|AB| = |A||B| = |B||A| = |BA|$$
\end{theorem}

\begin{theorem}[Binet $-$ Cauchy 公式]
    设 $A = (a_{ij})_{s\times n},B = (b_{ij})_{n\times m}$,则

    1. 如果 $s>n$,那么 $|AB| = 0$;

    2. 如果 $s\leqslant n$,那么  $|AB|$ 等于 $A$ 的所有 $s$ 阶子式的与相应 $s$ 阶子式的乘积之和,即
    $$|AB|=\sum_{1\leqslant v_1<\cdots< v_s\leqslant n}A \dbinom{1,\cdots,s}{v_1,\cdots,v_s} \cdot B \dbinom{v_1,\cdots,v_s}{1,\cdots,s}$$
\end{theorem}

\begin{definition}
    对于数域 $K$ 上的矩阵 $A$,如果存在数域 $K$ 上的矩阵 $B$,使得
    $$AB = BA = I$$
    那么称 $A$ 是可逆矩阵(或非奇异矩阵),$B$ 称为 $A$ 的逆矩阵,记作 $A^{-1}$。
\end{definition}

易知可逆矩阵一定是方阵,$n$ 级矩阵 $A$ 可逆的充分必要条件是
$$|A| \ne 0$$

\begin{definition}
    设矩阵 $A = (a_{ij})$,那么 $A$ 的伴随矩阵为
    $$A^*=(A_{ij})$$
\end{definition}

有
$$AA^* = |A|I$$

\begin{theorem}
    数域 $K$ 上 $n$ 级矩阵 $A$ 可逆的充分必要条件是 $|A| \ne 0$。当 $A$ 可逆时,
    $$A^{-1} = \frac{A^*}{|A|}$$
\end{theorem}

易得,可逆矩阵有如下性质

1. $(A^{-1})^{-1} = A$

2. $(AB)^{-1} = B^{-1}A^{-1}$

可逆矩阵能够通过初等行变换变成第简化行阶梯形矩阵一定是单位矩阵。

\begin{theorem}
    矩阵 $A$ 可逆的充分必要条件是它可以表示成一些初等矩阵的乘积。
\end{theorem}

用一个可逆矩阵左(右)乘一个矩阵 $A$,不改变 $A$ 的秩。

这里给出了求可逆矩阵第逆矩阵的又一种方法,称为初等变换法。

\section{矩阵的分块}

由矩阵 $A$ 的若干行、若干列的交叉位置元素按原来顺序排成的矩阵称为 $A$ 的一个子矩阵。若把分为若干组,列也分成若干组,从而 $A$ 被分成若干个子矩阵,把 $A$ 看成是由这些子矩阵组成的,这称为矩阵的分块,这种由子矩阵组成的矩阵称为分块矩阵。

\begin{theorem}
    设 $A = \left(\begin{matrix} B & C\\ 0 & D \end{matrix}\right)$,其中 $B,C,D$ 都是方阵,那么
    $$A^{-1} = \left(\begin{matrix} B^{-1} & -B^{-1}CD^{-1}\\ 0 & D^{-1} \end{matrix}\right)$$
\end{theorem}

\section{正交矩阵}

\begin{definition}
    实数域上的 $n$ 级矩阵 $A$ 如果满足
    $$AA'=I$$
    那么称 $A$ 是正交矩阵。
\end{definition}

那么其具有如下性质

1. 若 $A$ 和  $B$ 都是 $n$ 级正交矩阵,则 $AB$ 也是正交矩阵。

2. 若 $A$ 是正交矩阵,则 $A^{-1}$ (即 $A'$)也是正交矩阵。

3. 若 $A$ 是正交矩阵,则 $|A|=\pm 1$


引用 Kronecker 记号 $\delta_{ij}$,它的含义是
$$\delta_{ij}=\begin{cases}
    1,\quad \text{当}\ i=j\\
    0,\quad \text{当}\ i\ne j
\end{cases}$$

\begin{theorem}
    设实数域上 $n$ 级矩阵 $A$ 的行向量组为 $\ggg_1,\cdots,\ggg_n$,列向量组为 $\aaa_1,\cdots,\aaa_n$,则
    $$\ggg_i\ggg_j'=\delta_{ij},\aaa_i'\aaa_j=\delta_{ij},1 \leqslant i,j \leqslant n$$
\end{theorem}

\begin{definition}
    在 $\RR^n$ 中,任给 $\aaa = (a_1,\cdots,a_n),\bbb=(b_1,\cdots,b_n)$,规定
    $$(\aaa,\bbb) := \sum a_nb_n = \aaa\bbb'$$
    这个二元实值函数 $(\aaa,\bbb)$ 称为 $\RR^n$ 的一个内积(通常称为标准内积)。
\end{definition}

可以验证 $\RR^n$ 的标准内积有下列性质:

1. 对称性 $(\aaa,\bbb) = (\bbb,\aaa)$。

2. 线性性之一 $(\aaa+\ggg,\bbb) = (\aaa,\bbb) + (\ggg,\bbb)$。

3. 线性性之二 $(k\aaa,\bbb) = k(\aaa,\bbb)$。

4. 正定性 $(\aaa,\aaa)\geqslant 0$,当且仅当 $\aaa=\ling$ 时等号成立。

可以验证
$$(k_1\aaa_1+\k_2\aaa_2,\bbb) = k_1(\aaa_1,\bbb) + k_2(\aaa_2,\bbb)$$
$$(\aaa,k_1\bbb_1+k_2\bbb_2) = k_1(\aaa,\bbb_1) + k_2(\aaa,\bbb_2)$$

$n$ 维向量空间 $\RR^n$ 有了标准内积后,就称 $\RR^n$ 为一个欧几里得空间。其中,向量 $\aaa$ 的长度 $|\aaa|$ 规定为
$$|\aaa| := \sqrt{(\aaa,\aaa)}$$

容易验证
$$|k\aaa| = |k||\aaa|$$

长度为 $1$ 的向量称为单位向量,把非零向量 $\aaa$ 除以 $|\aaa|$ 称为把 $\aaa$ 单位化。

在欧几里得空间 $\RR^n$ 中,如果 $(\aaa,\bbb)=0$,那么称 $\aaa$ 与 $\bbb$ 是正交的,记作 $\aaa \bot \bbb$。由非零向量组成的向量组如果其中每两个不同的向量都正交,那么称它们为正交向量组。如果其每个向量都是单位向量,那么称它为正交单位向量组。

特殊的,零向量与任何向量正交,仅由一个非零向量组成的向量组也是正交向量组。

容易验证,欧几里得空间 $\RR^n$ 中 $n$ 个向量组成的正交向量组一定是 $\RR^n$ 的一个基,称它为正交基。如果其每个向量都是单位向量,那么称它为 $\RR^n$ 的一个标准正交基。

\begin{theorem}
    实数域上的 $n$ 级矩阵 $A$ 是正交矩阵的充分必要条件为:$A$ 的行(列)向量组是欧几里得空间 $\RR^n$ 的一个标准正交基。
\end{theorem}

\begin{theorem}
    设 $\aaa_1,\cdots,\aaa_s$ 是欧几里得空间 $\RR^n$ 中一个线性无关的向量组,令
    \begin{equation*}
        \begin{aligned}
            \bbb_1&=\aaa_1\\
            \bbb_2&=\aaa_2 - \frac{\aaa_2,\bbb_1}{\bbb_1,\bbb_1}\bbb_1\\
            &\cdots \\
            \bbb_s &= \aaa_s-\sum_{j=1}^{s-1}\frac{\aaa_s,\bbb_j}{\bbb_j,\bbb_j}\bbb_j
        \end{aligned}
    \end{equation*}
    则 $\bbb_1,\cdots,\bbb_s$ 是正交向量组,并且与 $\aaa_1,\cdots,\aaa_s$ 等价。
\end{theorem}

这给出了在欧几里得空间 $\RR^n$ 中从一个线性无关的向量组 $\aaa_1,\cdots,\aaa_s$ 出发,够造出与它等价的一个正交向量组的方法,这种方法称为 施密特(Schmidt)正交化过程,只要再将 $\bbb_1,\cdots,\bbb_s$ 中每个向量单位化,则其就是与原向量组等价的正交单位向量组,就是 $\RR^n$ 的一个标准正交基。

\section{\texorpdfstring{$K^n$ 到 $K^s$ 的线性映射}{Kn 到 Ks 的线性映射}}

映射视作熟知的。设 $f$ 是集合 $S$ 到集合 $S'$,$S$ 所有元素在 $f$ 下的象组成的集合叫做 $f$ 的值域或者 $f$ 的象,记作 $f(S)$ 或 $\Im f$。

\begin{definition}
    数域 $K$ 上的向量空间 $K^n$ 到 $K^s$ 的一个映射 $\sigma$ 如果保持加法和数量乘法,即 $\forall \aaa,\bbb \in K^n,k\in K$ 有
    $$\sigma (\aaa+\bbb) = \sigma(\aaa) + \sigma(\bbb)$$
    $$\sigma(k\aaa) = k\sigma(\aaa)$$
    那么称 $\sigma$ 是 $K^n$ 到 $K^s$ 上的一个线性映射。
\end{definition}

设 $A$ 是数域 $K$ 上 $s\times n$ 矩阵,令
\begin{equation*}
    \begin{aligned}
        \AAA : K^n &\longrightarrow K^s\\
        \aaa &\longmapsto A\aaa
    \end{aligned}
\end{equation*}


\begin{definition}
    设 $\sigma$ 是 $K^n$ 到 $K^s$ 的一个映射,$K^n$ 的一个子集
    $$\{\aaa \in K^n \mid \sigma(\aaa) = \ling\}$$
    称为映射 $\sigma$ 的核,记作 $\ker \sigma$
\end{definition}

容易验证,如果 $\sigma$ 是 $K^n$ 到 $K^s$ 的一个线性映射,那么 $\ker \sigma$ 是 $K^n$ 的一个子空间。

\begin{theorem}
    设数域上 $K$ 上齐次线性方程组 $A\XXX = \ling$ 的解空间是 $W$,$A$ 对应的线性映射为 $\AAA$ 则
    $$\ker \AAA = W$$
\end{theorem}

因此
$$\dim \ker \AAA + \dim \Im \AAA = \dim K^n$$
 % 函数的连续性
\chapter{矩阵的相抵与相似}

\section{等价关系与集合的划分}

等价关系还是记录一下。

\begin{definition}
    设 $S$ 是一个非空集合,我们把 $S\times S$ 的一个子集 $W$ 叫做 $S$ 上的一个二元关系。如果 $(a,b)\in W$,那么称 $a$ 与 $b$ 有 $W$ 关系;如果 $(a,b)\notin W$,那么称 $a$ 与 $b$ 没有 $W$ 关系。
\end{definition}

当 $a$ 与 $b$ 有 $W$ 关系时,记作 $aWb$,或 $a\sim b$。

\begin{definition}
    集合 $S$ 上的一个二元关系 $\sim$ 乳沟具有下述性质:$\forall a,b,c\in S$,有

    1. 反身性 $a\sim a$

    2. 对称性 $a\sim b \Rightarrow b\sim a$

    3. 传递性 $a\sim b\ \text{且}\ b\sim c \Rightarrow a\sim c$ 

    那么称 $\sim$ 是 $S$ 上的一个等价关系。
\end{definition}

\begin{definition}
    设 $\sim$ 是集合 $S$ 上的一个等价关系,$a\in S$ ,令
    $$\overline{a} := \{x\in S \mid x\in a\}$$
    称 $\overline{a}$ 是由 $a$ 确定的等价类,$a$ 称为等价类 $\overline{a}$ 的一个代表。
\end{definition}

\begin{definition}
    如果集合 $S$ 是一些非空子集 $S_i$ ($i\in I$,这里 $I$ 表示指标集)的并集,并且其中不相等的子集一定不相交,那么称集合 $\{S
    _i \mid i\in I\}$ 是 $S$ 的一个划分,记作 $\pi(S)$。
\end{definition}

\begin{theorem}
    设 $\sim$ 是集合 $S$ 上的一个等价关系,则所有等价类组成的集合是 $S$ 的一个划分,记作 $\pi_\sim(S)$。
\end{theorem}

\begin{definition}
    设 $\sim$ 是集合 $S$ 上的一个等价关系。由所有等价类组成的集合称为 $S$ 对于关系 $\sim$ 的商集,记作 $S/\sim$。
\end{definition}

注意,$S$ 的商集 $S/\sim$ 里的元素是 $S$ 的子集,不是 $S$ 的元素。

设 $\sim$ 是集合 $S$ 上的一个等价关系,一种量或一种表达式如果对于同一个等价类里的元素是相等的,那么称这种量或表达式是一个不变量。恰好能完全决定等价类的一组不变量称为完全不变量。

\section{矩阵的相抵}

数域 $K$ 上所有 $s\times n$ 矩阵组成的集合记作 $M_{s\times n}(K)$,当 $s=n$ 时简记为 $M_n(K)$。

\begin{definition}
    对于数域 $K$ 上的 $s\times n$ 矩阵 $A$ 和 $B$,如果从 $A$ 经过一系列初等行变换和初等列变换能变成矩阵 $B$,那么称 $A$ 与 $B$ 是相抵的,记作 $A\overset{\text{相抵}}{\sim}B$。
\end{definition}

相抵是集合 $M_{s\times n}(K)$ 上的一个二元关系,容易验证相抵是一个等价关系,其下矩阵 $A$ 的等价类称为 $A$ 的相抵类。

\begin{theorem}
    设数域 $K$ 上 $s\times n$ 矩阵 $A$ 的秩为 $r$,如果 $r>0$,那么 $A$ 相抵于下述形式的矩阵
    $$\left(\begin{matrix}
        I_r & 0\\
        0   & 0
    \end{matrix}\right)$$
    称其为 $A$ 的相抵标准形;如果 $r=0$,那么相抵标准形是零矩阵。
\end{theorem}

\begin{theorem}
    数域 $K$ 上 $s\times n$ 矩阵 $A$ 与 $B$ 相抵当且仅当它们的秩相等,即矩阵的秩是相抵关系下的完全不变量。
\end{theorem}

\section{广义逆矩阵}

\begin{theorem}
    设 $A$ 是数域 $K$ 上 $s\times n$ 非零矩阵,则矩阵方程
    $$AXA = A$$
    一定有解。如果 $\rank(A) = r$,并且
    $$A = P\left(\begin{matrix}
        I_r & 0\\
        0   & 0
    \end{matrix}\right)Q$$
    其中 $P,Q$ 分别是 $K$ 上 $s$ 级、$n$ 级可逆矩阵,那么矩阵方程的通解为
    $$X = Q^{-1}\left(\begin{matrix}
        I_r & B\\
        C   & D
    \end{matrix}\right)P^{-1}$$
    其中 $B,C,D$ 分别是数域 $K$ 上任意的 $r\times (s-r),(n-r)\times r,(n-r)\times (s-r)$ 矩阵。
\end{theorem}

\begin{definition}
    设 $A$ 是数域 $K$ 上 $s\times n$ 非零矩阵,则矩阵方程 $AXA = A$ 的每一个解都称为 $A$ 的一个广义逆矩阵,简称广义逆。用 $A^-$ 表示。
\end{definition}

\begin{theorem}
    非齐次线性方程组 $A\XXX = \bbb$ 有解的充分必要条件是
    $$\bbb = AA^-\bbb$$
\end{theorem}

\begin{theorem}
    非齐次线性方程组 $A\XXX = \bbb$ 有解时,它的通解为
    $$\XXX  = A^-\bbb$$
\end{theorem}

\begin{theorem}
    数域 $K$ 上 $n$ 元齐次线性方程组 $A\XXX = 0$ 的通解为
    $$\XXX  = (I_n - A^-A)\boldsymbol{Z}$$
    其中 $A^-$ 是 $A$ 的任意一个广义逆,$\boldsymbol{Z}$ 取遍 $K^n$ 中任意列向量。
\end{theorem}

% \begin{definition}
%     设 $A$ 是复数域上 $s\times n$ 矩阵,矩阵方程组
%     $$\begin{cases}
%         AXA = A,\\
%         XAX = X,\\
%         (AX)^* = AX,\\
%         (XA)^* = XA,
%     \end{cases}$$
%     称为 $A$ 的 Penrose 方程组,它的解称为 $A$ 的 Moore-Penrose 广义逆,记作 $A^+$。
% \end{definition}44

\section{矩阵的相似}

\begin{definition}
    设 $A$ 与 $B$ 都是数域 $K$ 上 $n$ 级矩阵,如果存在数域 $K$ 上一个 $n$ 级可逆矩阵 $P$,使得
    $$P^{-1}AP = B$$
    那么称 $A$ 与 $B$ 是相似的,记作 $A\sim B$。
\end{definition}

同样相似是集合 $M_{s\times n}(K)$ 上的一个二元关系,容易验证相似是一个等价关系,其下矩阵 $A$ 的等价类称为 $A$ 的相似类。

相似的矩阵具有相等的行列式和秩。

\begin{definition}
    $n$ 级矩阵 $A=(a_{ij})$ 的主对角线上元素的称为 $A$ 的迹,记作 $\tr(A)$。
\end{definition}

容易验证矩阵的迹都有如下性质
\begin{equation*}
    \begin{aligned}
        \tr(A+B) &= \tr(A) + \tr(B)\\
         \tr(kA) &= k\tr(A)\\
         \tr(AB) &= \tr(BA)
    \end{aligned}
\end{equation*}

\begin{theorem}
    相似的矩阵都有相同的迹。
\end{theorem}

这表明,矩阵的行列式、秩、迹都是相似关系下的不变量,简称为相似不变量。

如果 $n$ 级矩阵相似于一个对角矩阵,那么称 $A$ 可对角化。

\begin{theorem}
    数域 $K$ 上 $n$ 级矩阵 $A$ 可对角化的充分必要条件是,$K^n$ 中有 $n$ 个线性无关的列向量 $\aaa_1,\cdots,\aaa_n$,以及 $K$ 中有 $n$ 个数 $\lambda_1,\cdots,\lambda_n$(它们之中有些可能相等),使得
    $$A\aaa_i = \lambda_i \aaa_i, i=1,\cdots,n$$
    这时,令 $P = (\aaa_1,\cdots,\aaa_n)$,则
    $$P^{-1}AP = \diag\{\lambda_1,\cdots,\lambda_n\}$$
\end{theorem}

\section{矩阵的特征值和特征向量}

\begin{definition}
    设 $A$ 是数域 $K$ 上的一个 $n$ 级矩阵,如果 $K^n$ 中有非零列向量 $\aaa$ 使得
    $$A \aaa = \lambda_0\aaa,\text{且}\ \lambda_0\in K$$
    那么称 $\lambda_0$ 是 $A$ 的一个特征值,称 $\aaa$ 是 $A$ 的属于特征值 $\lambda_0$ 的一个特征向量。
\end{definition}

注意零向量不是特征向量。

\begin{theorem}
    设 $A$ 是数域 $K$ 上的 $n$ 级矩阵,则

    1. $\lambda_0$ 是 $A$ 的一个特征值当且仅当 $\lambda_0$ 是 $A$ 的特征多项式 $|\lambda I-A|$ 在 $K$ 中的一个根。

    2. $\aaa$ 是 $A$ 的属于特征值 $\lambda_0$ 的一个特征向量当且仅当 $\aaa$ 是齐次线性方程组 $(\lambda_0I-A)\XXX = \ling$  的一个解。
\end{theorem}

设 $\lambda_j$ 是 $A$ 的一个特征值,把齐次线性方程组 $(\lambda_jI-A)\XXX = \ling$ 的解空间称为 $A$ 的属于 $\lambda_j$ 的特征子空间,其中的全部非零向量都是 $A$ 的属于 $\lambda_j$ 的全部特征向量。

\begin{theorem}
    相似的矩阵有相等的特征多项式。
\end{theorem}

因此矩阵的特征多项式和特征值都是相似不变量。

\begin{definition}
    设 $A$ 是数域 $K$ 上的 $n$ 级矩阵,$\lambda_1$ 是 $A$ 的一个特征值。把 $A$ 的属于 $\lambda_1$ 的特征子空间的维数叫做特征值 $\lambda_1$ 的几何重数,而把 $\lambda_1$ 作为 $A$ 的特征多项式根的重数叫做 $\lambda_1$ 的代数重数。
\end{definition}

代数重数简称为重数。

\begin{theorem}
    设 $\lambda_1$ 是数域 $K$ 上的 $n$ 级矩阵 $A$ 的一个特征值,则 $\lambda_1$ 的几何重数不超过它的代数重数。
\end{theorem}

\section{矩阵可对角化的条件}

\begin{theorem}
    数域 $K$ 上 $n$ 级矩阵 $A$ 可对角化的充分必要条件是 $A$ 有 $n$ 个线性无关的特征向量 $\aaa_1,\cdots,\aaa_n$,此时令
    $$P=(\aaa_1,\cdots,\aaa_n)$$
    则
    $$P^{-1}AP = \diag\{\lambda_1,\cdots,\lambda_n\}$$
    其中 $\lambda_i$ 是 $\aaa_i$ 所属的特征值。上述对角矩阵称为 $A$ 的相似标准形。
\end{theorem}

\begin{theorem}
    设 $\lambda_1,\lambda_2$ 是数域 $K$ 上 $n$ 级矩阵 $A$ 的不同特征值,$\aaa_1,\cdots,\aaa_s$ 与 $\bbb_1,\cdots,\bbb_r$ 分别是 $A$ 的属于 $\lambda_1,\lambda_2$ 的线性无关的特征向量,则 $\aaa_1,\cdots,\aaa_s,\bbb_1,\cdots,\bbb_r$ 线性无关。
\end{theorem}

\begin{theorem}
    设 $\lambda_1,\cdots,\lambda_m$ 是数域 $K$ 上 $n$ 级矩阵 $A$ 的不同特征值,$\aaa_{j1},\cdots,\aaa_{jr_j}$ 是 $A$ 的属于 $\lambda_j$ 的线性无关的特征向量,$j=1,\cdots,m$,则向量组
    $$\aaa_{11},\cdots,\aaa_{1r_1},\cdots,\aaa_{m1},\cdots,\aaa_{1r_m}$$
    线性无关。
\end{theorem}

\begin{theorem}
    数域 $K$ 上 $n$ 级矩阵 $A$ 可对角化的充分必要条件是:$A$ 的特征多项式的全部复根都属于 $K$,并且 $A$ 的每个特征值的几何重数等于它的代数重数。
\end{theorem}

\section{实对称矩阵的对角化}

若对于 $n$ 级矩阵 $A,B$,存在一个 $n$ 级正交矩阵 $T$,使得 $T^{-1}AT=B$,那么称 $A$ 正交相似于 $B$。

\begin{theorem}
    实对称矩阵的特征多项式的每一个复根都是实数,从而它们都是特征值。
\end{theorem}

\begin{theorem}
    实对称矩阵 $A$ 的属于不同特征值的特征向量是正交的。
\end{theorem}

\begin{theorem}
    实对称矩阵一定正交相似于对角矩阵。
\end{theorem}

对于 $n$ 级实对称矩阵 $A$,找一个正交矩阵 $T$,使得 $T^{-1}AT$ 为对角矩阵的步骤如下。

1. 计算 $|\lambda I- A|$,求出它的全部不同的根:$\lambda_1,\cdots,\lambda_m$,它们是 $A$ 的特征值。

2. 对于每一个特征值 $\lambda_j$,求 $(\lambda_jI-A)\XXX = \ling$ 的一个基础解系 $\aaa_{j1},\cdots,\aaa_{jr_j}$;然后把它们施密特正交化和单位化,得到 $\eee_{j1},\cdots,\eee_{jr_j}$。它们也是 $A$ 的属于 $\lambda_j$ 的一个特征向量。

3. 令
$$T=(\eee_{11},\cdots,\eee_{1r_1},\cdots,\eee_{m1},\cdots,\eee_{mr_m})$$
则 $T$ 是 $n$ 级正交矩阵,且
$$T^{-1}AT = \diag\{\lambda_{1},\cdots,\lambda_{1},\cdots,\lambda_{m},\cdots,\lambda_{m}\}$$ % 导数和微分

\end{document}