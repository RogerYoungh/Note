\chapter{函数的连续性}

\section{连续性的概念}

\begin{definition}[连续性]
	设函数 $f$ 在某 $U(x_0)$ 上有定义。若
	$$\lim_{x\to x_0}f(x) = f(x_0)$$
	则称 $f$ 在点 $x_0$ 连续。
\end{definition}

记 $\Delta x = x-x_0$,称为自变量 $x$ 在点 $x_0$ 的增量或改变量。设 $y_0=f(x_0)$,相应的函数 $y$ 在点 $x_0$ 的增量记为
$$\Delta y = f(x)-f(x) = f(x+\Delta)-f(x_0) = y-y_0$$ 

连续性的 $\varepsilon-\delta$ 形式定义:若对任给的 $\varepsilon>0$,存在 $\delta>0$,使得当 $|x-x_0|<\delta$ 时,有 $|f(x)-f(x_0)|<\varepsilon$,则称函数 $f$ 在点 $x_0$ 连续。

或者进一步表示为
$$\lim_{x\to x_0}f(x) = f\left(\lim_{x\to x_0}x\right)$$

\begin{definition}
	设函数 $f$ 在某 $U_+(x_0)$ 上有定义。若
	$$\lim_{x\to x_0^+}f(x) = f(x_0)$$
	则称 $f$ 在点 $x_0$ 右连续。同理左连续。
\end{definition}

因此函数 $f$ 在点 $x_0$ 连续的充要条件是:$f$ 在点 $x_0$ 既是左连续,又是右连续。

\begin{definition}[间断点]
	设函数 $f$ 在某 $U^\circ(x_0)$ 上有定义。若 $f$ 在点 $x_0$ 无定义,或 $f$ 在点 $x_0$ 有定义而不连续,则称点 $x_0$ 为函数 $f$ 的间断点或不连续点。
\end{definition}

若 $\displaystyle\lim_{x\to x_0}f(x)=A$,而 $f$ 在点 $x_0$ 无定义,或有定义但 $f(x_0)\ne A$,则称点 $x_0$ 为 $f$ 的可去间断点。

若函数 $f$ 在点 $x_0$ 的左、右极限都存在,但 $\displaystyle\lim_{x\to x_0^+}f(x) \ne \lim_{x\to x_0^-}f(x)$,则称点 $x_0$ 为函数 $f$ 的跳跃间断点。

可去间断点与跳跃间断点统称为第一类间断点,所有其他形式的间断点统称为第二类间断点。

若函数 $f$ 在区间 $I$ 上的每一点都连续,则称 $f$ 为 $I$ 上的连续函数。对于闭区间或半开区间的端点,函数在这些点上连续是指左连续或右连续。

\section{连续函数的性质}

\begin{theorem}[局部有界性]
	若函数 $f$ 在点 $x_0$ 连续,则 $f$ 在某 $U(x_0)$ 上有界。
\end{theorem}

\begin{theorem}[局部保号性]
	若函数 $f$ 在点 $x_0$ 连续,且 $f(x_0)>0$,则对任何正数 $r<f(x_0)$,存在某 $U(x_0)$,使得对一切 $x\in U(x_0)$,有 $f(x)>r$。
\end{theorem}

\begin{theorem}[四则运算]
	若函数 $f,g$ 在点 $x_0$ 连续,则 $f\pm g,f\cdot g,f/g$ 也都在点 $x_0$ 连续。
\end{theorem}

\begin{theorem}
	若函数 $f$ 在点 $x_0$ 连续,$g$ 在点 $u_0$ 连续,$u_0=f(x_0)$,则复合函数 $g\circ f$ 在 $x_0$ 连续。
\end{theorem}

\begin{definition}
	设 $f$ 为定义在数集 $D$ 上的函数。若存在 $x_0\in D$,使得对一切 $x\in D$,有 $f(x_0)\ge f(x)$,则称 $f$ 在 $D$ 上有最大值,并称 $f(x_0)$ 为 $f$ 在 $D$ 上的最大值。
\end{definition}

\begin{theorem}[最大、最小值定理]
	若函数 $f$ 在闭区间 $[a,b]$ 上连续,则 $f$ 在闭区间 $[a,b]$ 上有最大值与最小值。
\end{theorem}

\begin{theorem}[介值定理]
	若函数 $f$ 在闭区间 $[a,b]$ 上连续,且 $f(a)\ne f(b)$。若 $\mu$ 为介于 $f(a)$ 和 $f(b)$ 之间的任何实数。则至少存在一点 $x_0\in (a,b)$ 使得 $f(x_0)=\mu$。
\end{theorem}

\begin{theorem}
	若函数 $f$ 在 $[a,b]$ 上严格单调并连续,则反函数 $f^{-1}$ 在其定义域 $[\min\{f(a),f(b)\},\max\{f(a),f(b)\}]$ 上连续。
\end{theorem}

\begin{definition}
	设 $f$ 是定义在区间 $I$ 上的函数。若对任给的 $\varepsilon>0$,存在 $\delta=\delta(\varepsilon)>0$,使得对任何 $x',x''\in I$,只要 $|x'-x''|<\delta$,就有 $|f(x')-f(x'')|<\varepsilon$,就称函数 $f$ 在去间 $I$ 上一致连续。
\end{definition}

\begin{theorem}[一致连续性]
	若函数 $f$ 在闭区间 $[a,b]$ 上连续,则 $f$ 在 $[a,b]$ 上一致连续。
\end{theorem}

\section{初等函数的连续性}

\begin{theorem}
	设 $p>0$,$a,b$ 为任意两个实数,则有
	$$p^a\cdot p^b = p^{a+b},(p^a)^b=p^{ab}$$
\end{theorem}

\begin{theorem}
	指数函数 $a^x(a>0)$ 在 $\RR$ 上是连续的。
\end{theorem}
