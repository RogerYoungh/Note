\chapter{数列极限}

\section{数列极限的概念}

\begin{definition}[数列极限的 $\varepsilon - N$ 定义]
	设 $\{a_n\}$ 为数列,$A$ 为定数。若对任给的正数 $\varepsilon$,总存在正整数 $N=N(\varepsilon)$,使得当 $n>N$ 时有
	$$|a_n - A| < \varepsilon$$
	则称数列 $\{a_n\}$ 收敛于 $A$,或称 $A$ 为数列 $\{a_n\}$ 的极限,记作
	$$\displaystyle\lim_{n\to \infty} a_n = A \text{,或}\ a_n \to a(n \to \infty)$$
\end{definition}

等价定义:任给 $\varepsilon > 0$,若在 $U(A;\varepsilon)$ 之外数列 $\{a_n\}$ 中的项至多只有有限个,则称 $\{a_n\}$ 收敛于极限 $A$。

若对于数列 $\{a_n\}$,不存在 $A$ 使得 $a_n\to A$,则称数列 $\{a_n\}$ 发散。

特殊地,若 $\displaystyle\lim_{n\to \infty} a_n = 0$,则称 $\{a_n\}$ 为无穷小数列。

\begin{definition}[无穷大数列]
	若数列 $\{a_n\}$ 满足:对任意正数 $M>0$,存在正整数 $N$,使得当 $n>N$ 时,

	1. $a_n>M$,则称数列 $\{a_n\}$ 发散于正无穷大,记作 $\displaystyle\lim_{n\to \infty} a_n = +\infty$,或 $a_n \to +\infty$。

	2. 有 $a_n<M$,则称数列 $\{a_n\}$ 发散于负无穷大,记作 $\displaystyle\lim_{n\to \infty} a_n = -\infty$,或 $a_n \to -\infty$。
\end{definition}

\section{收敛数列的性质}

\begin{theorem}[唯一性]
	若数列 $\{a_n\}$ 收敛,则它只有一个极限。
\end{theorem}

\begin{proof}
	如果数列 $\{a_n\}$ 同时以 $A,B$ 为极限,即任给 $\varepsilon>0$,总存在 $N_1,N_2$,使得
	$$|a_n-A|<\varepsilon,n>N_1;\quad |a_n-B|<\varepsilon,n>N_2$$
	那么当 $n>\max\{N_1,N_2\}$ 时需要恒成立
	$$2\varepsilon > |a_n-A|+|a_n-B| \geqslant |A-B|$$
	当 $A\ne B$ 时,对于 $2\varepsilon <|A-B|$ 不恒成立,因此只能 $A=B$。
\end{proof}

\begin{theorem}[有界性]
	若数列 $\{a_n\}$ 收敛,则 $\{a_n\}$ 有界。
\end{theorem}

\begin{proof}
	不妨设 $\displaystyle\lim_{n\to \infty} a_n = A$。令 $\varepsilon = 1$,那么存在 $n>N$ 使得
	$$|a_n-A| \leqslant 1$$
	令
	$$M = \{|a_1|,\cdots,|a_N|,|A-1|,|A+1|\}$$
	那么对任意正整数 $n$,总有 $|a_n|\leqslant M$。
\end{proof}

\begin{theorem}[保号性]
	若 $\displaystyle\lim_{n\to \infty}a_n = a>0$,则对任何 $a'\in (0,a)$ 存在正数 $N$,使得当 $n>N$ 时,有 $a_n>a'$。
\end{theorem}

\begin{theorem}[保不等式性,保序性]
	设 $\displaystyle\lim_{n\to \infty} a_n = A,\displaystyle\lim_{n\to \infty} b_n = B$,则有

	(1)如果存在 $n>N$ 使得 $a_n\geqslant b_n$ 恒成立,则 $A\geqslant B$。

	(2)反之,如果 $A>B$,则存在 $n>N_1$ 使得 $a_n>b_n$ 恒成立。
\end{theorem}

\begin{theorem}[迫敛性,夹逼定理]
	设数列 $\{a_n\},\{b_n\},\{c_n\}$ 满足当 $n>N_0$ 有 $a_n\leqslant c_n\leqslant b_n$。若
	$$\lim_{n\to \infty}a_n = A = \lim_{n\to \infty}c_n$$
	则 $\displaystyle\lim_{n\to \infty}b_n = A$。
\end{theorem}

\begin{proof}
	即对于任给的 $\varepsilon>0$,存在 $N_1,N_2$,使得当 $n>N_1$ 有 
	$$A-\varepsilon<a_n<A+\varepsilon$$
	当 $n>N_2$ 有 
	$$A-\varepsilon<c_n<A+\varepsilon$$
	因此当 $n>\max\{N_0,N_1,N_2\}$ 时,有
	$$A-\varepsilon < a_n \leqslant b_n \leqslant c_n < A+\varepsilon$$
\end{proof}

\begin{theorem}[四则运算法则]
	若 $\{a_n\}$ 与 $\{b_n\}$ 均为收敛数列。则
	\begin{equation*}
		\begin{aligned}
			\lim_{n\to \infty}(a_n \pm b_n) &= \lim_{n\to \infty}a_n \pm \lim_{n\to \infty}b_n\\
			\lim_{n\to \infty}(a_n \cdot b_n) &= \lim_{n\to \infty}a_n \cdot \lim_{n\to \infty}b_n
		\end{aligned}
	\end{equation*}
	若再假设 $b_n \ne 0$ 及 $\displaystyle\lim_{n\to \infty}b_n \ne 0$,则
	\begin{equation*}
		\lim_{n\to \infty}\frac{a_n}{b_n} = \frac{\lim_{n\to \infty}a_n}{\lim_{n\to \infty}b_n}
	\end{equation*}
\end{theorem}

\begin{definition}[数列的子列]
	设 $\{a_n\}$ 为数列,$\{n_k\}$ 为正整数集 $\NN^+$ 的无限子集,且 $n_1<n_2<\cdots<n_k<\cdots$,则数列 $\{a_{n_k}\}$ 称为数列 $\{a_n\}$ 的一个子列。
\end{definition}

\begin{theorem}
	数列 $\{a_n\}$ 收敛的充要条件:$\{a_n\}$ 的任何子列都收敛。
\end{theorem}

\section{数列极限存在的条件}

\begin{definition}
	若数列 $\{a_n\}$ 各项满足关系式 $a_n \leqslant a_{n+1}(a_n \geqslant a_{n+1})$,则称 $\{a_n\}$ 为递增(递减)数列,统称为单调数列。
\end{definition}

\begin{theorem}[单调有界定理]
	有界的单调数列必有极限。
\end{theorem}

\begin{theorem}[致密性定理]
	任何有界数列必定有收敛的子列。
\end{theorem}

\begin{theorem}[柯西(Cauchy)收敛准则]
	数列 $\{a_n\}$ 收敛的充要条件是:对任给的 $\varepsilon >0$,存在正整数 $N$,使得当 $n,m>N$ 时,有 
	$$|a_n-a_m| < \varepsilon$$
\end{theorem}

柯西收敛准则的条件称为柯西条件。

\section{Stolz 公式}

\begin{theorem}
	对于任意的 $1 \leqslant k \leqslant n$,设 $b_k>0$ 且 $m \leqslant \dfrac{a_k}{b_k} \leqslant M$,则有
	$$m \leqslant \frac{\sum a_n}{\sum b_n} \leqslant M$$
\end{theorem}

\begin{theorem}[Stolz 公式一]
	设数列 $\{x_n\},\{y_n\}$,且 $\{y_n\}$ 严格单调地趋于 $+\infty$,如果
	$$\lim_{n\to \infty}\frac{x_n-x_{n-1}}{y_n-y_{n-1}}=A$$
	则
	$$\lim_{n\to \infty} \frac{x_n}{y_n} = A$$
\end{theorem}

\begin{proof}
	分类讨论 Todo……
\end{proof}

\begin{theorem}[Stolz 公式二]
	设数列 $\{y_n\}$ 严格单调地趋于 $0$,且数列 $\{x_n\}$ 也收敛到 $0$,那么如果
	$$\lim_{n\to \infty}\frac{x_n-x_{n-1}}{y_n-y_{n-1}}=A$$
	则
	$$\lim_{n\to \infty} \frac{x_n}{y_n} = A$$
\end{theorem}

\begin{proof}
	分类讨论 Todo……
\end{proof}

\section{例题}

\begin{problem}
	设 $\displaystyle\lim_{n\to\infty}a_n=A$,求证:$\displaystyle\lim_{n\to\infty}\frac{\sum a_n}{n}=A$。
\end{problem}

\begin{solution}
	即对于任给的 $\varepsilon>0$,存在 $n>N_1$ 使得
	$$|a_n-A|<\dfrac{\varepsilon}{2}$$
	那么变形有
	$$\left|\frac{\sum a_n}{n}-A\right| \leqslant \frac{\sum |a_n-A|}{n} = \frac{\sum_{k=1}^{N_1} |a_k-A|}{n} + \frac{\sum_{k=N_1+1}^{n} |a_k-A|}{n}$$
	注意到 $\sum_{k=1}^{N_1} |a_k-A|$ 已经为定值,从而存在 $n>N_2$ 使得
	$$\frac{\sum_{k=1}^{N_1}|x_k-A|}{n}<\frac{\varepsilon}{2}$$
	因此当 $n>\max\{N_1,N_2\}$ 时有
	$$LHS < \frac{\varepsilon}{2}+\frac{n-N_1}{n}\times \frac{\varepsilon}{2} < \frac{\varepsilon}{2}+\frac{\varepsilon}{2} = \varepsilon$$
\end{solution}