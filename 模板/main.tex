\documentclass[cn,11pt,blue,normal,founder]{elegantbook}
\usepackage{amsmath}
\usepackage{amsfonts}
\usepackage{amssymb}
\usepackage[cache=false]{minted}

\begin{document}
\newcommand\mfrac[2]{\dfrac{#1\smash[b]{\strut}}{#2\smash[t]{\strut}}}
\newcommand\ee{\mathrm{e}}
\newcommand\dd{\mathrm{d}}
\newcommand\uppi{\mathrm{\pi}}
%%----------- 封面部分 ----------- %%

\begin{titlepage}
	\vspace*{35ex}% A4->35,Boox->25,Kindle->10
	\begin{minipage}{.9\textwidth}
	\flushright
		{\zihao{0}\textbf{ACM 模板}}\\%%中文书名
		\rule{\linewidth}{1pt}\\ \vspace{2ex}
		{\zihao{2}\textsf{ACM Template}} \\%% 英文书名
		\vspace{30ex}% A4->30,Boox->20,Kindle->10
		{\zihao{4}rogeryoungh}%% 作者
	\end{minipage}
	\vfill\centering
	{\zihao{4}\number\year 年 \number\month 月 \number\day 日 · \LaTeX{}}
\end{titlepage}
%%----------- 目录部分 ----------- %%
\clearpage{\hypersetup{hidelinks}\tableofcontents}

%%----------- 主体部分 ----------- %%
\clearpage

\chapter{上号}

\section{头文件}

\lstinputlisting[style=cpp]{上号/头文件.h}

\section{预编译}

头文件引入方式改为如下,可以把头文件放入 \lstinline[style=cpp]{lab.h} ,然后使用 \lstinline[language=bash]{g++ lab.h} 预编译。

实际编译使用 \lstinline[language=bash]{g++ lab.cpp -D YCYLOCAL} 添加条件编译参数。

\begin{lstlisting}[style=cpp]
#ifdef YCYLOCAL
#include "lab.h"
#else  //#include <bits/stdc++.h>
#include <list>      #include <stack>      #include <iostream>
#include <cmath>     #include <cstdio>     #include <algorithm>
#include <queue>     #include <cstring>    #include <functional>
#endif
\end{lstlisting}

\section{进制转换}

\lstinputlisting[style=cpp]{上号/进制转换.cpp}

\section{常见技巧}

向上取整 $p/q$ 为 \lstinline[style=cpp]{(p-1)/q+1} 。

%\lstinputlisting[style=cpp]{上号/二分查找.cpp}

%\section{二分查找}

%\lstinputlisting[style=cpp]{上号/二分查找.cpp}

\section{快速幂} %洛谷P1226

\lstinputlisting[style=cpp]{上号/快速幂.cpp}

\section{矩阵快速幂}

\lstinputlisting[style=cpp]{上号/矩阵快速幂.cpp}

\section{快速排序}

\lstinputlisting[style=cpp]{上号/快速排序.cpp}

\section{第 $k$ 大数}

\lstinputlisting[style=cpp]{上号/第k大数.cpp}

\section{链表}

\lstinputlisting[style=cpp]{上号/链表.cpp}


\chapter{数论}

\section{GCD 和 LCM} %洛谷P1888

\lstinputlisting[style=cpp]{数论/gcdlcm.cpp}

\section{EXGCD} %洛谷P5656

对于方程 $ax+by=c$,令 $g=\gcd(a,b)$,若 $g\nmid c$ 则无解。

因此可通过 $exgcd$ 先求出 $ax+by=g$ 的整数解,再转换回原方程的解。



\lstinputlisting[style=cpp]{数论/exgcd.cpp}

\section{Eratosthenes 筛}

\lstinputlisting[style=cpp]{数论/Eratosthenes筛.cpp}

\section{Eular 筛}

\lstinputlisting[style=cpp]{数论/Eular筛.cpp}

\section{素性测试}

\subsection{试除法}

\lstinputlisting[style=cpp]{数论/试除法.cpp}

\subsection{Miller Rabbin}

如果 $n\leqslant 2^{32}$,那么 $ppp$ 取 $2,7,61$;如果 $ppp$ 选择 $2,3,7,61,24251$,那么 $10^{16}$ 内只有唯一的例外。如果莫名 WA 了,就多取点素数吧。

%如果 $n\leqslant 2^{32}$,那么 ppp 取 $2,7,61$;如果 ppp 选择 $2,3,7,61,24251$,那么 $10^16$ 内只有唯一的 $46856248255981$,若选择 $2,3,5,7,11,13,17$ ,则 $341550071728320$ 以内都是正确的。

\lstinputlisting[style=cpp]{数论/Miller_Rabbin.cpp}

\chapter{图论}

\section{链式前项星}

\lstinputlisting[style=cpp]{图论/链式前项星.cpp}

\section{Dijkstra}

\lstinputlisting[style=cpp]{图论/Dijkstra.cpp}

\section{Bellman-Ford}

\lstinputlisting[style=cpp]{图论/Bellman-Ford.cpp}

\section{Floyd}

起始条件 $f(i,j)=edge(i,j)$,$f(i,i)=0$。

\lstinputlisting[style=cpp]{图论/Floyd.cpp}

\section{最近公共祖先(LCA)}

如果数据小,可以不用求 $\log_2$,直接莽 $20$。

\lstinputlisting[style=cpp]{图论/LCA.cpp}


\chapter{动态规划}

\section{背包}

\subsection{01 背包}

给定体积为 $v_i$,价值 $w_i$ 的 $N$ 个物品,背包容积为 $M$,每个物品只能取 $1$ 个,求最大价值。

\lstinputlisting[style=cpp]{动态规划/01背包.cpp}

\subsection{完全背包}

给定体积为 $v_i$,价值 $w_i$ 的 $n$ 个物品,背包容积为 $v$,每个物品任意取,求最大价值。

\lstinputlisting[style=cpp]{动态规划/完全背包.cpp}

\subsection{多重背包}

给定体积为 $v_i$,价值 $w_i$ 的 $N$ 个物品,背包容积为 $M$,每个物品有 $c_i$ 个,求最大价值。

如各种背包组合(如洛谷 P1833 樱花),通常把完全背包转为 $99999$ 个(适当调节)多重背包,再按 01 背包来。

\lstinputlisting[style=cpp]{动态规划/多重背包.cpp}

\section{最长公共上升序列}

给出 $1,2,\ldots,n$ 的两个排列 $a$ 和 $b$ ,求它们的最长公共子序列。

\lstinputlisting[style=cpp]{动态规划/最长公共上升序列.cpp}

\section{数字计数}

试计算在区间 $1$ 到 $n$ 的所有整数中,数字 $x(0 \leqslant x \leqslant 9)$共出现了多少次?

\lstinputlisting[style=cpp]{动态规划/数字计数.cpp}




\end{document}
